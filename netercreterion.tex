\documentclass[12pt]{article}
\usepackage[utf8]{inputenc}
\usepackage[russian]{babel}
\usepackage{amssymb,amsmath}
\usepackage[noend]{algorithmic}
\usepackage{tikz}  
\usepackage{tkz-berge}
\usepackage{multicol}
\usepackage{ntheorem}
\usepackage{lipsum}
\usetikzlibrary{graphs,arrows,shapes,positioning}
\usepackage{graphicx}
\textheight=24cm % высота текста
\textwidth=16cm % ширина текста
\oddsidemargin=0pt % отступ от левого края
\topmargin=-1.5cm % отступ от верхнего края
\parindent=24pt % абзацный отступ
\parskip=0pt % интервал между абзацами
\tolerance=2000 % терпимость к "жидким" строкам
\flushbottom % выравнивание высоты страниц
%\def\baselinestretch{1.5} % печать с большим интервалом

\theoremstyle{break}

\newtheorem{Defini}{Определение}
\newtheorem{Th}{Теорема}
\newtheorem{Con}{Следствие}
\newenvironment{Proof} % имя окружения
	{\par\noindent{\bf Доказательство.}} % команды для \begin
	{\hfill$\scriptstyle\blacksquare$} % команды для \end
\def\P{\mathcal{P}}
\def\S{$S(X_n)$}
\def\K{\mathcal{K}_m}
\def\B{$\mathcal{B}$}
\def\S{S_n}
\def\algorithmicrequire{\textbf{Вход:}}
\def\algorithmicensure{\textbf{Выход:}}

\newcommand{\LD}{\langle}
\newcommand{\RD}{\rangle}

\bibliographystyle{plain} %TODO добавить к проекту gost.bst файл для оформления библиографии по ГОСТу

\title{Критерий нётеровости по уравнениям для частичных порядков}
\author{И.Д. Кудык, А.Ю. Никитин}
%\date{}

\begin{document}
    \maketitle % вывести заголовок, автора, дату
    \thispagestyle{empty} % не нумеровать первую страницу
    \tableofcontents % сгенерировать оглавление
    \newpage
    
    \section{Введение}
    Одной из старейших задач в математике является решение уравнений и систем уравнений с различными коэффициентами. Классическая алгебраическая геометрия изучает решения систем уравнений (алгебраические множества) над полями вещественных и комплексных чисел. Период большого развития классической алгебраической геометрии приходится на 19 век. Но с середины 20 века и по сей день большое развитие происходится на универсальную алгебраическую геометрию. Универсальная алгебраическая геометрия изучает алгебраические множества над произвольными алгебраическими системами, буть то граф, группа или универсальная алгебра. Фиксированным результатом развития универсальной алгебраической геометрии на текущий момент может служить монография Э.Ю.~Данияровой, А.Г.~Мясникова и В.Н.~Ремесленникова \cite{AlgGeom}. В настоящей статье будет изучаться алгебраическая геометрия над частичными порядками. И, если теория частичных порядков уже получила серьезное развитие \cite{Gretzer}, то вопросы алгебраической геометрии ещё не решены для данных математических объектов. Отдельные классы частичных порядков, таких как решетки, полурешетки и булевы алгебры, уже изучены с алгебро-геометрической точки зрения, например в \cite{Semilattice}. Было рассмотрено строение координатных алгебр, неприводимых алгебраических множеств, исследовано свойство нётеровости по уравнениям и компактности для частичных порядков, а, так же, аксиоматизируемость классов частичных порядков. Свойство нётеровости по уравнениям для алгебраической структуры является очень важным свойством, которое позволяет применять множество результатов из алгебраической геометрии. В настоящей статье будет изучено свойство нётеровости по уравнениям для произвольного частичного порядка и, как результат, будет сформулирован критерий нётеровости по уравнениям для произвольного частичного порядка и будут приведены примеры использования данного критерия.

	\section{Предварительные сведения}
		В данной секции мы напомним базовые определения из теории частичных порядков и универсальной алгебраической геометрии над частичными порядками.
			
		\textit{Частично упорядоченным множеством (частичным порядком)} называется алгебраическая система $\P = \langle P | \leqslant^{(2)}, A\rangle,$ где $\leqslant$ -- предикатный символ отношения порядка и $A$ -- множество константных символов, на которой выполнены 3 аксиомы:
		\begin{enumerate}
			\item $\forall p\in P\ p\leqslant p.$
			\item $\forall p_1, p_2\in P\ p_1 \leqslant p_2 \wedge p_2 \leqslant p_1 \rightarrow p_1=p_2.$
			\item $\forall p_1, p_2, p_3\in P\ p_1 \leqslant p_2 \wedge p_2 \leqslant p_3 \rightarrow p_1 \leqslant p_3.$
		\end{enumerate}

		Язык частичного порядка с константами будем обозначать через $L_A.$
		
		Элементы $x$ и $y$ частичного порядка $\P$ называются \textit{сравнимыми}, если $x\leqslant y$ или $y\leqslant x.$ В противном случае, элементы называются \textit{несравнимыми}.
	
		Для любого множества элементов $A$ частичного порядка $\P$ определим множества $A^{\uparrow} = \{ x\in\P\ |\ \forall a\in A\ a\leqslant x\}$ и $A^{\downarrow} = \{ a\in\P\ |\ \forall a\in A\ x\leqslant a\}$. Эти множества называются \textit{верхним и нижним конусами} множества $A$ (или просто верхнее и нижнее множества А). Для одноэлементного множества $A=\{a\}$ обозначения, соответственно, будут $a^{\uparrow}$ и $a^{\downarrow}.$

		Множеством \textit{порождающих} элементов конуса $A^{\uparrow}(A^{\downarrow})$ называется множество элементов $B$ частичного порядка $\P$, для которого верно $B^{\uparrow}=A^{\uparrow}~(B^{\downarrow}=A^{\downarrow})$. Конус $A^{\uparrow}~(A^{\downarrow})$ называется \textit{конечно порожденным}, если существует конечное множество $B$ порождающих этого конуса.

		Пусть задано множество элементов частичного порядка $A$. \textit{Верхним A-конусом} называется пара $(A, A^{\uparrow})$, которая состоит из \textit{базы} $A$ и верхнего конуса $A^{\uparrow}$, порожденного базой $A$. Аналогично определяется нижний $A$-конус. Верхний $A$-конус называется конечно порожденным, если существует такое конечное подмножество элементов $B\subseteq A$, что $B^{\uparrow} = A^{\uparrow}$.

		Пусть $X = \{x_1,\dots,x_n\}$ -- множество переменных. \textit{Термом} языка $L_A$ частичного порядка $\P$ от переменных $X$ является любая константа из множества константных символов $A$ или любая переменная из множества $X$. \textit{Атомарная формула} языка $L_A$ частичного порядка $\P$ от переменных $X$ определяется следующим образом:
		\begin{enumerate}
			\item если $t_1, t_2$ -- термы, то $t_1=t_2$ -- атомарная формула;
			\item если $t_1, t_2$ -- термы, то $t_1\leqslant t_2$ -- атомарная формула.
		\end{enumerate}

		\textit{Уравнением} над частичным порядком $\P$ в языке $L_A$ от переменных $X$ мы будем называть любую атомарную формулу над $\P$. Выпишем в явном виде все типы уравнений над частичными порядками:
		\begin{enumerate}
			\item $a_i=a_j$, где $a_i, a_j\in A$,
			\item $a_i\leqslant a_j$, где $a_i, a_j\in A$,
			\item $x_i=a_j$, где $x_i\in X,~a_j\in A$,
			\item $x_i=x_j$, где $x_i, x_j\in X$,
			\item $a_i\leqslant x_j$, где $x_j\in X,~a_i\in A$,
			\item $x_i\leqslant a_j$, где $x_i\in X,~a_j\in A$,
			\item $x_i\leqslant x_j$, где $x_i, x_j\in X$.
		\end{enumerate}

		\textit{Система уравнений} $S(X)$ от переменных $X=\{x_1,\dots,x_n\}$ -- это любое множество уравнений от переменных $X$. Точка $p=(p_1,\dots,p_n)\in P^n$ называется \textit{решением} системы уравнений $S(X)$, если при подстановке в каждое уравнение $\varphi(x_1,\dots,x_n)$ системы $S(X)$ вместо переменных соответствующих координат точки $p$, получится верное над $\P$ выражение: $\P\vDash \varphi(p_1,\dots,p_n)$. Естественным образом определяется \textit{множество решений} системы уравнений. Множество решений системы $S(X_n)$ будем обозначать как $V(S)$. Две системы уравнений $S_1$ и $S_2$ называются \textit{эквивалентными}, если множества их решений совпадают (обозначается как $S_1\sim S_2$).

		Договоримся обозначать множество уравнений типа $a_i=a_j$, где $a_i, a_j\in A$, в системе уравнений $S(X_n)$ через $S_{a=a}$. Аналогично обозначаются множества остальных шести типов уравнений $S_{a\leqslant a}, S_{x=a}, S_{x=x}, S_{a\leqslant x}, S_{x\leqslant a}$ и $ S_{x\leqslant x}$ для системы уравнений $S(X_n)$.
		
		Говорят, что частичный порядок $\P$ обладает \textit{свойством нётеровости по уравнениям}, если любая система уравнений $S(X_n)$ от $n$ переменных над $\P$ имеет эквивалентную ей конечную подсистему $S'(X_n) \subseteq S(X_n)$. Если же любой системе уравнений над $\P$ эквивалентна какая-нибудь конечная система уравнений над $\P$, то $\P$ обладает свойством \textit{слабой нётеровости по уравнениям}.
		
		Будем говорить, что класс частичных порядков $\mathcal{K}$ обладает свойством (слабой) нётеровости по уравнениям, если любой частичный порядок $\P \in \mathcal{K}$ обладает свойством (слабой) нётеровости по уравнениям.
		
	
	\section{Основной результат}
		В данной секции будет сформулирован и доказан основной результат статьи: критерий нётеровости по уравнениям для произвольного частичного порядка.

		\begin{Th}\label{criterion}
			Частичный порядок $\P$ обладает свойством нётеровости по уравнениям тогда и только тогда, когда для любого подмножества $A$ элементов частичного порядка $\P$ верхний и нижний $A$-конусы являются конечно порожденными.
		\end{Th}

		\begin{Proof}
			Сначала, пусть для частичного порядка $\P$ любые $A$-конусы являются конечно порожденными. Пусть задана система уравнений $S(X_n)$ над $\P$ в языке $L_A$ от $n$ переменных. Сразу предположим, что эта система содержит все 7 типов уравнений и уравнений каждого типа бесконечное число.

			Без ограничения общности, считаем, что все различные константы в системе $S(X_n)$ имеют различные интерпретации на $\P$. Это условие не является обязательным, но упростит дальнейшие рассуждения и не повлияет на суть доказательства.

			Требуется доказать, что из системы $S(X_n)$ можно выбрать конечную подсистему $S'(X_n)$, эквивалентную  изначальной системе. Заметим, что изначальная система $S(X_n)$ разбивается на 7 непересекающихся подсистем $S_{a=a}, S_{x=a}$ и т.д. Множество решений системы $S(X_n)$ можно определить следующим образом: $V(X) = V(S_{a=a})\cap V(S_{a\leqslant a})\cap V(S_{x=a})\cap V(S_{x=x})\cap V(S_{x\leqslant a})\cap V(S_{a\leqslant x})\cap V(S_{x\leqslant x})$. Покажем, что из подсистем каждого типа уравнений можно выбрать конечную подсистему, эквивалентную изначальному подмножеству. То есть, например, из подсистемы $S_{x=x}\subset S(X_n)$ можно выбрать конечную подсистему $S'_{x=x}$, эквивалентную $S_{x=x}$.

			Сначала рассмотрим подмножество системы уравнений $S_{a=a}\subset S(X_n)$. Если это множество уравнений совместно над $\P$, то оно никак не влияет на множество решений системы уравнений и подсистема $S'_{a=a} = \emptyset$. Если же $\P\nvDash S_{a=a},$ тогда существует уравнение $a_i=a_j\in S_{a=a},$ которое не выполнено над частичным порядком $\P$ и это уравнение будет эквивалентно подсистеме $S_{a=a}$. Аналогичные рассуждения применимы к множеству уравнений $S_{a\leqslant a}\subset S(X_n).$

			Ввиду конечности множества переменных, можно выбрать конечные подмножества уравнений $S'_{x=x}$ и $S'_{x\leqslant x}$ , эквивалентных подсистемам $S_{x=x}$ и $S_{x\leqslant x}$ соответственно. Отметим, что системы $S_{x=x}$ и $S_{x\leqslant x}$ не могут быть противоречивыми.

			Если множество уравнений $S_{x=a}\subset S(X_n)$ совместно, то, опять, в виду конечности множества переменных, существует конечная подсистема $S'_{x=a}\subset S_{x=a}$, эквивалентная $S_{x=a}.$ Иначе, в $S_{x=a}$  существует такая пара уравнений $x_i=a_j, x_i=a_k,$ что $a_j\neq a_k$. Эта пара уравнений и будет эквивалентной $S_{x=a}$ подсистемой.

			Отметим, что до сих пор мы не воспользовались условиями из формулировки критерия. Осталось рассмотреть подмножества уравнений $S_{x\leqslant a}\subset S(X_n)$ и $S_{a\leqslant x}\subset S(X_n)$. Сначала рассмотрим $S_{x\leqslant a}$.

			Выделим подмножество уравнений $S_{x_i\leqslant a}\subset S_{x\leqslant a}$, в котором все уравнения зависят от переменной $x_i$. Это множество можно представить следующим образом: $S_{x_i\leqslant a}=\{x_i\leqslant a_j\ |\ j\in J, |J| = \infty\}.$ Из этого представления видно, что $V(S_{x_i\leqslant a}) = A_i^{\downarrow}$, где $A_i = \{a_j\ |\ j\in J\}$. Пользуясь свойством, что любые $A$-конусы частичного порядка $\P$ конечно порожденные, можно выбрать такое конечное подмножество $J'\subset J$, что $A'_i = \{a_j\ |\ j\in J'\}$ и $(A'_i)^{\downarrow} = A_i^{\downarrow}.$ Это означает, что можно выбрать конечную подсистему $S'_{x_i\leqslant a}\sim S_{x_i\leqslant a}$.

			Применяя описанную процедуру для всех переменных из $S_{x\leqslant a}$, можно выделить конечную подсистему $S'_{x\leqslant a}\sim S_{x\leqslant a}.$ Аналогичными рассуждениями можно вывести существование конечной подсистемы $S'_{a\leqslant x}\sim S_{a\leqslant x}.$

			Были построены конечные подсистемы $S'_{a=a}, S'_{a\leqslant a}, S'_{x=a}, S'_{x=x}, S'_{a\leqslant x}, S'_{x\leqslant a}$ и $S'_{x\leqslant x}$, эквивалентные подсистемам $S_{a=a}, S_{a\leqslant a}, S_{x=a}, S_{x=x}, S_{a\leqslant x}, S_{x\leqslant a}$ и $S_{x\leqslant x}$ системы $S(X_n)$, соответственно. Это означает, что $V(X) = V(S'_{a=a})\cap V(S'_{a\leqslant a})\cap V(S'_{x=a})\cap V(S'_{x=x})\cap V(S'_{x\leqslant a})\cap V(S'_{a\leqslant x})\cap V(S'_{x\leqslant x})$.
			%Тем самым показано, что если в частичном порядке $\P$ любой $A$-конус является конечно порожденным, то для любой бесконечной системы $S(X_n)$ найдется эквивалентная ей конечная подсистема $S'_{x=x}\cup S'_{x\leqslant x}\cup S'_{x=a}\cup S'_{x\leqslant a}\cup S'_{a\leqslant x}.$  

			Теперь, пусть частичный порядок $\P$ нётеров по уравнениям. Выберем произвольное бесконечное множество $A$ частичного порядка $\P$. Рассмотрим его нижний конус $A^{\downarrow}$. Составим систему уравнений $S(x) = \{x\leqslant a_i\ |\ a_i\in A\}$ над $\P$. Ввиду того, что $\P$ нётеров по уравнениям, можно выбрать конечную подсистему $S'(x)=\{x\leqslant a_i\ |\ a_i\in A', |A'|<\infty\}\sim S(x)$. Это означает, что $A^{\downarrow} = A'^{\downarrow},$ что доказывает конечную порожденность нижнего $A$-конуса для любого множества элементов $A$ частичного порядка. Аналогично доказывается конечная порожденность верхнего $A$-конуса.

		\end{Proof}

		Отметим, что если $A$-конус не является конечно порожденным, то это не означает, что $A^{\uparrow}$ или $A^{\downarrow}$ не являются конечно порожденными. Рассмотрим следующий частичный порядок: $\P = \mathbb{Z}\cup \{q\},$ где $\mathbb{Z}$ -- линейный порядок на множестве целых чисел, а $q$ -- единичный элемент, не сравнимый ни с одним элементом из $\mathbb{Z}.$ Схема данного частичного порядка изображена на рис.~\ref{simpleNeter}.

		\begin{figure}[h]
%			\hfill
			\begin{center}
				\begin{tikzpicture}[node distance = 4 cm]
					%\SetGraphUnit{5}

					\node (d1) at (0, 1) {.};
					\node (d2) at (0, 0.7) {.};
					\node (d3) at (0, 0.4) {.};
					\node (p1) at (0,0) {$i-1$}; 
				    \node (p2) at (0,-2) {$i$};
				    \node (p3) at (0,-4) {$i+1$};
				    \node (d4) at (0, -4.4) {.};
					\node (d5) at (0, -4.7) {.};
					\node (d6) at (0, -5) {.};

				    \node (q) at (2, -2) {q};
				    \node at (1.8, -2) {.};

				    \begin{scope}
				       \draw (p1) -- (p2);
				       \draw (p2) -- (p3);
				    \end{scope}
				\end{tikzpicture}
				%\hfill
			\end{center}
			\caption{Частичный порядок $\P = \mathbb{Z}\cup \{q\}$}
			\label{simpleNeter}
		\end{figure}

		Рассмотрим нижний $\mathbb{Z}$-конус. Видно, что $\mathbb{Z}^{\downarrow} = \emptyset$. Любое конечное подмножество $\mathbb{Z}$ образует непустой нижний конус. Это означает, что нижний $\mathbb{Z}$-конус не является конечно порожденным. Но если взять множество $B = \{0, q\}$, то $B^{\downarrow} = \emptyset = \mathbb{Z}^{\downarrow}$, что означает конечную порожденность конуса $\mathbb{Z}^{\downarrow}$.

		Ввиду этого замечания и доказанного критерия нётеровости по уравнениям для частичных порядков, можно сформулировать следующий критерий.

		\begin{Con}
			Частичный порядок $\P$ обладает свойством слабой нётеровости по уравнениям тогда и только тогда, когда для любого подмножества $A$ элементов частичного порядка $\P$ конусы $A^{\uparrow}$ и $A^{\downarrow}$ являются конечно порожденными.
		\end{Con}

		Из критерия нётеровости по уравнениям для частичного порядка легко получить критерий нётеровости по уравнениям для класса частичных порядков. Класс частичных порядков является нётеровым по уравнениям тогда и только тогда, когда любой частичный порядок из этого класса является нётеровым по уравнениям. Из этого следует, что класс частичных порядков является нётеровым по уравнениям тогда и только тогда, когда для любого множества элементов $A$ частичного порядка верхние и нижние $A$-конусы являются конечно порожденными конусами.


	\section{Применение критерия нётеровости по уравнениям для частичных порядков}
		В данной секции будет показан ряд примеров применения критерия нётеровости по уравнениям к различным классам частичных порядков.

		1. Рассмотрим класс всех частичных порядков $\mathcal{K}_{\P}.$ Категория всех частичных порядков содержит бесконечный линейный порядок $\mathbb{Z}$ на целых числах. Рассмотрим $\mathbb{Z}$-конус. Как было показано ранее, $\mathbb{Z}^{\uparrow} = \emptyset$. Любое конечное подмножество $Z\subset\mathbb{Z}$ имеет непустой верхний конус. Значит, $\mathbb{Z}$-конус не является конечно порожденным. Следовательно, класс $\mathcal{K}_{\P}$ по теореме \ref{criterion} не является нётеровым по уравнениям.

		2. Рассмотрим класс двудольных частичных порядков $\mathcal{K}_2$. \textit{Двудольным частичным порядком} $\P$ будем называть всякий частичный порядок, который состоит из двух непересекающихся множеств элементов $A$ и $B$, где элементы из множества $B$ больше элементов из множества $A$, а любые элементы из $A$ не сравнимы между собой и любые элементы из $B$ не сравнимы между собой.

		Этот класс так же не является нётеровым по уравнениям. Достаточно рассмотреть частичный порядок $\mathcal{N}$, у которого каждый элемент $b_j$ из множества $B = \{b_i\ |\ i\in \mathbb{N}\cup\{0\}\}$ сравним с каждым элементом множества $A = \{a_i\ |\ i\in \mathbb{N}\cup\{0\}\} \setminus \{a_j\}$. Схема данного частичного порядка показана на рис.~\ref{bipartPoset}.

		\begin{figure}[h]
			\begin{center}
				\begin{tikzpicture}[node distance = 4 cm]
					\begin{scope}%[rotate=90]
						\SetVertexMath
						\grEmptyLadder[RA=2,RB=4]{4}   
					\end{scope}
					\Edges(a0,b1)\Edges(a0,b2)\Edges(a0,b3)
					\Edges(a1,b0)\Edges(a1,b2)\Edges(a1,b3)
					\Edges(a2,b0)\Edges(a2,b1)\Edges(a2,b3)
					\Edges(a3,b0)\Edges(a3,b1)\Edges(a3,b2)
								 %(x, y)
					\node (d1) at (6.8, 4) {.};
					\node (d2) at (7.1, 4) {.};
					\node (d3) at (7.4, 4) {.};
					\node (d4) at (6.8, 0) {.};
					\node (d5) at (7.1, 0) {.};
					\node (d6) at (7.4, 0) {.};
				\end{tikzpicture}
				\caption{Схема частичного порядка $\mathcal{N}$}
				\label{bipartPoset}
			\end{center}
		\end{figure}

		Если взять множество $A_0 = \{a_i~|~i\in \mathbb{N}\}$, то $A_0^{\uparrow} = \{b_0\}$. Но любое конечное подмножество $A_0$ будет порождать отличный от $A_0^{\uparrow}$ конус. Следовательно, по теореме~\ref{criterion}, класс $\mathcal{K}_2$ не является нётеровым по уравнениям.

		3. Рассмотрим класс частичных порядков $\K$ с ограниченными верхними конусами. Для любого элемента $p$ частичного порядка из $\K$ мощность верхнего конуса $p^{\uparrow}$ ограничено натуральным числом $m$. Сначала докажем конечную порожденность любого нижнего $A$-конуса.

		Докажем конечную порожденность любого нижнего конуса в классе $\K$. Пусть зафиксировано множество элементов частичного порядка $A$ и нижний конус $A^{\downarrow}$ является бесконечно порожденным. Это означает, что $|A| = \infty$. Если $A^{\downarrow} = \emptyset$, то существует подмножество $B\subset A$, мощность которого $|B| = m+1$ и $B^{\downarrow} \neq \emptyset$. Это означает, что $\forall b\in B^{\downarrow}~B\in b^{\uparrow}$ и, следовательно, $|b^{\uparrow}| \geqslant m+1$, что противоречит определению класса $\K$. Если $A^{\downarrow}\neq\emptyset$, то $\forall a\in A^{\downarrow}~A\in a^{\uparrow}$. Мощность множества $A$ больше $m$, что опять приводит к противоречию задания класса $K_m$. Поэтому любой нижний конус является конечно порожденным.

		Докажем, что любой нижний $A$-конус является конечно порожденным. Если множество $A$ -- конечное, то нижний $A$-конус является конечно порожденным и утверждение доказано. Пусть множество $A$ -- бесконечное. Если $A^{\downarrow} \neq \emptyset$, то, опять, $\forall a\in A^{\downarrow}~A\in a^{\uparrow}$. Получилось противоречие с заданием класса $\K$. Это означает, что если $|A| = \infty$, то $A^{\downarrow} = \emptyset$. Выделим такое подмножество $B\subset A$, что выполнены следующие условия:
		\begin{enumerate}
			\item $|B| = m + 1$,
			\item $\forall b_1,\dots,b_{m+1}\in B, b_i\neq b_j$ при $i\neq j$,
			\item $B^{\downarrow} \neq \emptyset$.
		\end{enumerate}
		Если такого множества $B$ не существует (не выполнено третье условие), то нижний $A$-конус является конечно порожденным, так как $B^{\downarrow} = A^{\downarrow}$. Если такое $B$ найдется, то $\forall b\in B^{\downarrow}~B\in b^{\uparrow}$ и $|b^{\uparrow}| > m$, что противоречит определению класса $\K$. Поэтому, если $|A|=\infty$, то нижний $A$-конус является конечно порожденным. Следовательно, любые нижние $A$-конусы частичного порядка из класса $\K$ являются конечно порожденными.

		Теперь докажем конечную порожденность верхних $A$-конусов. По определению класса $\K$, верхний конус любого множества элементов частичного порядка конечен. Пусть задано произвольное бесконечное множество $A = \{a_1, a_2,\dots\}$ (если $A$ -- конечное, то конечная порожденность доказана). Мощность его верхнего конуса $|A^{\uparrow}| = k$, где $0\leqslant k\leqslant m$. Построим такое конечное подмножество $A_0\subset A$, что $A_0^{\uparrow} = A^{\uparrow}$.

		Построение будет заключаться в следующей процедуре. Сначала $A_0$ пусто. Рассматриваем элемент $a_1$ множества $A$. Включаем в $A_0$ элемент $a_1$. Мощность множества $A_0$ будет $|A_0^{\uparrow}| = k_1$, где $k\leqslant k_1\leqslant m$. Далее, рассматриваем элемент $a_2$ множества $A$. Если выполнено условие $a_2^{\uparrow} \cap A_0^{\uparrow} = A_0^{\uparrow}$, то $a_2$ не включается в $A_0$. Иначе включается и мощность $A_0$ будет $|A_0^{\uparrow}| = k_2$, где $k\leqslant k_2 < k_1$. Далее, аналогичным образом, рассматривается элемент $a_3$ из множества $A$ и т.д. Данная процедура построит конечное множество $A_0=\{a_{i_1},\dots, a_{i_l}\}$, где $a_j\in A$.

		Мощность $|A_0^{\uparrow}| = k_l$, где $k_l = k$. В виду того, что $\forall C\subset A~|C^{\uparrow}| \geqslant |A^{\uparrow}|$, число $k_l$ не может быть меньше $k$. По построению множества $A_0$ видно, что если $k_l = k$, то $A^{\uparrow}$ содержит больше $k$ элементов. Поэтому конусы $A^{\uparrow}$ и $A_0^{\uparrow}$ совпадают. Следовательно, любой верхний $A$-конус частичного порядка из $\K$ является конечно порожденным. По теореме \ref{criterion} это доказывает нётеровость по уравнениям класса $\K$.

		\bibliography{neterbibl}
	
\end{document}
