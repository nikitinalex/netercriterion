\documentclass[12pt]{article}
\usepackage[utf8]{inputenc}
\usepackage[russian]{babel}
\usepackage{amssymb,amsmath}
\usepackage[noend]{algorithmic}
\usepackage{tikz}  
\usepackage{tkz-berge}
\usepackage{multicol}
\usepackage{ntheorem}
\usepackage{lipsum}
\usetikzlibrary{graphs,arrows,shapes,positioning}
\usepackage{graphicx}
\textheight=24cm % высота текста
\textwidth=16cm % ширина текста
\oddsidemargin=0pt % отступ от левого края
\topmargin=-1.5cm % отступ от верхнего края
\parindent=24pt % абзацный отступ
\parskip=0pt % интервал между абзацами
\tolerance=2000 % терпимость к "жидким" строкам
\flushbottom % выравнивание высоты страниц
%\def\baselinestretch{1.5} % печать с большим интервалом

\theoremstyle{break}

\newtheorem{Defini}{Определение}
\newtheorem{Problem}{Задача}
\newtheorem{Th}{Теорема}
\newtheorem{Con}{Следствие}
\newtheorem{Lem}{Лемма}
\newenvironment{Proof} % имя окружения
	{\par\noindent{\bf Доказательство.}} % команды для \begin
	{\hfill$\scriptstyle\blacksquare$} % команды для \end
\def\P{\mathcal{P}}
\def\S{$S(X_n)$}
\def\K{\mathcal{K}_m}
\def\B{$\mathcal{B}$}
\def\S{S_n}
\def\algorithmicrequire{\textbf{Вход:}}
\def\algorithmicensure{\textbf{Выход:}}

\newcommand{\LD}{\langle}
\newcommand{\RD}{\rangle}

\title{Критерий нётеровости для категорий частичного порядка}
\author{И.Д. Кудык, А.Ю. Никитин}
%\date{}

\begin{document}
    \maketitle % вывести заголовок, автора, дату
    \thispagestyle{empty} % не нумеровать первую страницу
    \tableofcontents % сгенерировать оглавление
    \newpage
    
    \section{Введение} % первый раздел

	\section{Предварительные сведения}
			
		\textit{Частично упорядоченным множеством (частичным порядком)} называется алгебраическая система $\P = \langle P | \leqslant^{(2)}, A\rangle,$ где $\leqslant$ -- предикатный символ отношения порядка и $A$ -- множество константных символов, на которой выполнены 3 аксиомы:
		\begin{enumerate}
			\item $\forall p\in P\ p\leqslant p.$
			\item $\forall p_1, p_2\in P\ p_1 \leqslant p_2 \wedge p_2 \leqslant p_1 \rightarrow p_1=p_2.$
			\item $\forall p_1, p_2, p_3\in P\ p_1 \leqslant p_2 \wedge p_2 \leqslant p_3 \rightarrow p_1 \leqslant p_3.$
		\end{enumerate}
		
		Элементы $x$ и $y$ частичного порядка $\P$ называются \textit{сравнимыми}, если $(x\leqslant y)\vee (y\leqslant x).$ Иначе, элементы \textit{несравнимы}.

		Элемент $p$ частично упорядоченного множества $\P$ называют:
		\begin{itemize}
			\item \textit{максимальным}, если $\forall x \in \P \ \ p\leqslant x \rightarrow p = x,$
			\item \textit{минимальным},  если $\forall x \in \P \ \ x\leqslant p \rightarrow p = x$
		\end{itemize}
		Множество максимальных элементов частичного порядка $\P$ обозначают $\max(\P)$, минимум - $\min(\P)$.

	
		Для любого множества элементов $A$ частичного порядка $\P$ можно определить множества $A^{\uparrow} = \langle x\in\P\ |\ \forall a\in A\ a\leqslant x\rangle$ и $A^{\downarrow} = \langle a\in\P\ |\ \forall a\in A\ x\leqslant a\rangle$. Эти множества называются \textit{верхними и нижними конусами} множества $A$ (или просто верхние и нижние множества А). Для одноэлементного множества $A=\{a\}$ обозначения соответственно $a^{\uparrow}$ и $a^{\downarrow}.$

		Переходим к определениям уравнений над частичными порядками. \textit{Уравнение} над частичным порядком $\P = \langle P~|~\leqslant^{(2)}, A\rangle$ от переменных $X$ называется выражение одного из следующих типов:
		\begin{enumerate}
			\item $a_i=a_j$, где $a_i, a_j\in A$,
			\item $a_i\leqslant a_j$, где $a_i, a_j\in A$,
			\item $x_i=a_j$, где $x_i\in X,~a_j\in A$,
			\item $x_i=x_j$, где $x_i, x_j\in X$,
			\item $a_i\leqslant x_j$, где $x_j\in X,~a_i\in A$,
			\item $x_i\leqslant a_j$, где $x_i\in X,~a_j\in A$,
			\item $x_i\leqslant x_j$, где $x_i, x_j\in X$.
		\end{enumerate}

		\textit{Система уравнений} $S(X)$ от переменных $X=\{x_1,\dots,x_n\}$ -- это любое множество уравнений от переменных $X$. Точка $p=(p_1,\dots,p_n)\in P^n$ называется \textit{решением} системы уравнений $S(X)$, если при подстановке в любое уравнение системы $S(X)$ вместо переменных соответствующих координат точки $p$, получится верное над $\P$ выражение. Естественным образом определяется множество решений системы уравнений. Две системы уравнений $S_1$ и $S_2$ называются \textit{эквивалентными}, если множества их решений совпадают (обозначается как $S_1\sim S_2$).
		
		
		Говорят, что частичный порядок $\P$ обладает \textit{свойством нётеровости}, если любая система уравнений $S(X_n)$ от $n$ переменных над $\P$ имеет эквивалентную ей конечную подсистему $S'(X_n) \subseteq S(X_n)$. Если же любой системе уравнений над $\P$ эквивалентна какая-нибудь конечная система уравнений над $\P$, то $\P$ обладает свойством \textit{слабой нётеровости}.
		
		Категория частичных порядков $\mathcal{K}$ обладает свойством (слабой) нётеровости, если любой частичный порядок $\P \in \mathcal{K}$ обладает свойством (слабой) нётеровости.
		
	
	\section{Основной результат}
		Любое множество элементов $A$ частичного порядка $\P$ порождает верхний и нижний конус $A^{\uparrow}$ и $A^{\downarrow}$. Верхний конус множества $A$ называется \textit{конечнопорожденным}, если существует конечное множество элементов $B$ частичного порядка $\P$ такое, что $A^{\uparrow} = B^{\uparrow}$ (для нижнего конуса соответственно, $A^{\downarrow} = B^{\downarrow}$). Если, при этом, $B\subseteq A,$ то конус $A^{\uparrow}$ называется \textit{собственным}.

		\begin{Th}
			Категория частичных порядков $\mathcal{K}$ является нётеровой по уравнениям тогда и только тогда, когда для любого подмножества $A$ элементов частичного порядка из категории $\mathcal{K}$ конусы $A^{\uparrow}$ и $A^{\downarrow}$ являются собственными конечнопорожденными.
		\end{Th}

		\begin{Proof}
			Сначала, пусть для частичного порядка $\P$ из категории $\mathcal{K}$ конусы любого подмножества его элементов являются собственными конечнопорожденными. Пусть задана система уравнений $S(X_n)$ над $\P = \langle P~|~L_A\rangle$ от $n$ переменных. Сразу предположим, что эта система содержит все 7 типов уравнений и уравнений каждого типа тоже бесконечное число.

			Без ограничения общности, считаем, что все различные константы в системе $S(X_n)$ имеют различные интерпретации на $\P$. Это условие не является обязательным, но упростит дальнейшие рассуждения и не повлияет на суть доказательства.

			Сначала рассмотрим подмножество системы уравнений $S_{a=a}\subset S(X_n)$. Если это множество уравнений совместно над $\P$, то оно никак не влияет на множество решений системы уравнений. Если же $\P\nvDash S_{a=a},$ тогда существует уравнение $a_i=a_j\in S_{a=a},$ которое не выполнено над частичным порядком $\P$ и это уравнение будет эквивалентно системе $S(X_n).$ Аналогичные рассуждения применимы к множеству уравнений $S_{a\leqslant a}\subset S(X_n).$

			Ввиду конечности множества переменных, можно выбрать конечные подмножества уравнений $S'_{x=x}$ и $S'_{x\leqslant x}$ , эквивалентных системам $S_{x=x}$ и $S_{x\leqslant x}$ соответственно. Отметим, что системы $S_{x=x}$ и $S_{x\leqslant x}$ не могут быть противоречивыми.

			Если множество уравнений $S_{x=a}\subset S(X_n)$ совместно, то, опять, в виду конечности множества переменных, существует конечная подсистема $S'_{x=a}\subset S_{x=a}$, эквивалентная $S_{x=a}.$ Иначе, в $S_{x=a}$  существует такая пара уравнений $x_i=a_j, x_i=a_k,$ что $a_j\neq a_k$. Эта пара уравнений и будет эквивалентной $S(X_n)$ подсистемой.

			Отметим, что до сих пор мы не воспользовались условиями из формулировки критерия. Осталось рассмотреть подмножества уравнений $S_{x\leqslant a}\subset S(X_n)$ и $S_{a\leqslant x}\subset S_{X_n}$. Сначала рассмотрим $S_{x\leqslant a}$.

			Выделим подмножество уравнений, зависящих только от одной переменной $S_{x_i\leqslant a}\subset S_{x\leqslant a}$. Это множество можно представить следующим образом: $S_{x_i\leqslant a}=\{x_i\leqslant a_j\ |\ j\in J, |J| = \infty\}.$ Из этого представления видно, что решение системы уравнений $S_{x_i\leqslant a}$ принадлежит конусу $A_i^{\downarrow}$, где $A_i = \{a_j\ |\ j\in J\}$. Пользуясь свойством, что конусы для любого множества элементов частичного порядка собственные конечнопорожденные, можно выбрать конечное подмножество $J'\subset J$ такое, что $A'_i = \{a_j\ |\ j\in J'\}$ и $(A'_i)^{\downarrow} = A_i^{\downarrow}.$ Это означает, что можно выбрать конечную подсистему $S'_{x_i\leqslant a}\sim S_{x_i\leqslant a}$.

			Применяя описанную процедуру для всех переменных из $S_{x\leqslant a}$, можно выделить конечную подсистему $S'_{x\leqslant a}\sim S_{x\leqslant a}.$ Аналогичными рассуждениями можно вывести существование конечной подсистемы $S'_{a\leqslant x}\sim S_{a\leqslant x}.$

			Тем самым показано, что если верхний и нижний конусы любого подмножества элементов частичного порядка собственные конечнопорожденные, то для любой бесконечной системы $S(X_n)$ найдется эквивалентная ей конечная подсистема $S'_{x=x}\cup S'_{x\leqslant x}\cup S'_{x=a}\cup S'_{x\leqslant a}\cup S'_{a\leqslant x}.$  

			Теперь, пусть категория частичных порядков $\mathcal{K}$ является нётеровой по уравнениям. Выберем произвольное бесконечное множество $A$ частичного порядка $\P$ из категории $\mathcal{K}.$ Рассмотрим его нижний конус $A^{\downarrow}$. Составим систему уравнений $S(x) = \{x\leqslant a_i\ |\ a_i\in A\}$ над $\P$. Ввиду того, что категория $\mathcal{K}$ нётерова по уравнениям, можно выбрать конечную подсистему $S'(x)=\{x\leqslant a_i\ |\ a_i\in A', |A'|<\infty\}\sim S(x)$. Это означает, что $A^{\downarrow} = A'^{\downarrow},$ что доказывает собственную конечнопорожденность нижнего конуса для любого множества элементов частичного порядка. Аналогично доказывается собственная конечнопорожденность верхних конусов.

		\end{Proof}

		Отметим, что не любой конечнопорожденный конус является собственным. Рассмотрим следующий частичный порядок: $\P = \mathbb{Z}\cup \{q\},$ где $\mathbb{Z} = \{z_i\ |\ i\in I\}$ -- линейный порядок, а $q$ -- единичный элемент, не сравнимый ни с одним элементом из $\mathbb{Z}.$

		\begin{figure}[h]
%			\hfill
\begin{center}
			\begin{tikzpicture}[node distance = 4 cm]
%		  	\SetGraphUnit{5}

			\node (d1) at (0, 1) {.};
			\node (d2) at (0, 0.7) {.};
			\node (d3) at (0, 0.4) {.};
			\node (p1) at (0,0) {$z_{i-1}$}; 
		    \node (p2) at (0,-2) {$z_i$};
		    \node (p3) at (0,-4) {$z_{i+1}$};
		    \node (d4) at (0, -4.4) {.};
			\node (d5) at (0, -4.7) {.};
			\node (d6) at (0, -5) {.};

		    \node (q) at (2, -2) {q};
		    \node at (1.8, -2) {.};

		    \begin{scope}
		       \draw (p1) -- (p2);
		       \draw (p2) -- (p3);
		    \end{scope}
			\end{tikzpicture}
%			\hfill
\end{center}
			\caption{Частичный порядок $\P = \mathbb{Z}\cup \{q\}$}
			\label{figLeft}
		\end{figure}

		В качестве системы уравнений выберем бесконечную систему от одной переменной $S(x) = \{x\leqslant z_i\ |\ i\in I\}.$ Множество решений данной системы уравнений будет пустым. Любая конечная подсистема будет иметь не пустое множество решений. Это означает, что нижний конус $\mathbb{Z}^{\downarrow}$ не является собственным конечнопорожденным. Но если взять множество $B = \{z_0, q\}$, то $B^{\downarrow} = \emptyset$. Это означает, что конус $\mathbb{Z}^{\downarrow}$ является просто конечнопорожденным.

		Ввиду этого замечания и доказанного критерия нётеровости для категорий частичного порядка, можно сформулировать следующий критерий слабой нётеровости.

		\begin{Con}
			Категория частичных порядков $\mathcal{K}$ обладает свойством слабой нётеровости по уравнениям тогда и только тогда, когда для любого подмножества $A$ элементов частичного порядка из категории $\mathcal{K}$ конусы $A^{\uparrow}$ и $A^{\downarrow}$ являются конечнопорожденными.
		\end{Con}


	\section{Примеры классов частичных порядков}
		Рассмотрим применение данного критерия на конкретных категориях частичных порядков.

		1. Категория всех частичных порядков $\mathcal{K}_{\P}.$ Данная категория, очевидно, не является нётеровой по уравнениям.

		2. Категория двудольных частичных порядков $\mathcal{K}_2$. Двудольный частичный порядок $\P$ -- это такой частичный порядок, который представим как $\P = \max(\P)\cup~\min(\P)$. Эта категория так же не является нётеровой по уравнениям. Достаточно рассмотреть частичный порядок $\mathcal{N}$, у которого множество максимумов $\{b_i\ |\ i\geqslant 0\}$ сравнимо со всем множеством минимумов $\{a_i\ |\ i\geqslant 0\}$, кроме элементов с одинаковыми индексами. Схема данного частичного порядка показана на рис.\ref{bipartPoset}.

		\begin{figure}[h]
			\begin{center}
				\begin{tikzpicture}
					\begin{scope}%[rotate=90]
						\SetVertexMath
						\grEmptyLadder[RA=2,RB=4]{4}   
					\end{scope}
					\Edges(a0,b1)\Edges(a0,b2)\Edges(a0,b3)
					\Edges(a1,b0)\Edges(a1,b2)\Edges(a1,b3)
					\Edges(a2,b0)\Edges(a2,b1)\Edges(a2,b3)
					\Edges(a3,b0)\Edges(a3,b1)\Edges(a3,b2)
				\end{tikzpicture}
				\caption{Схема частичного порядка $\mathcal{N}$}
				\label{bipartPoset}
			\end{center}
		\end{figure}

		Если взять множество $A = \{a_i~|~i\geqslant 1\}$, то $A^{\uparrow} = \{b_0\}$. Но любое конечное подмножество $A$ будет порождать отличный от $A^{\uparrow}$ конус. Следовательно категория $\mathcal{K}_2$ не является нётеровой по уравнениям.

		3. Категория частичных порядков $\K$ с ограниченными верхними конусами. Для любого элемента $p$ частичного порядка из $\K$ мощность верхнего конуса $p^{\uparrow}$ ограничено натуральным числом $m$. Пусть зафиксировано множество элементов частичного порядка $A$. Если нижний конус $A^{\downarrow}$ является бесконечнопорожденным, то $|(A^{\downarrow})^{\uparrow}| = \infty$, что противоречит определению класса $\K$. Следовательно, $A^{\downarrow}$ будет конечнопорожденным. Далее, если $|A| = \infty$, то $A^{\downarrow} = \emptyset$, иначе $|(A^{\downarrow})^{\uparrow}| = \infty$. Поэтому, если $|A|=\infty$, то $A^{\downarrow} = \{a_i, a_j\}^{\downarrow}$, где $a_i, a_j\in A$ и эти элементы не сравнимы. Такие элементы найдутся, потому что иначе для какого-то элемента $a_k$ верхний конус будет мощности большей $m$. Следовательно, нижние конуса всех элементов $A$ частичного порядка собственные конечнопорожденные.

		По определению класса, верхний конус любого множества элементов частичного порядка конечен. Построим множество $A_0\subset A$. Пусть дано множество $A = \{a_1, a_2,\dots\}.$ Включаем в $A_0$ элемент $a_1$. Мощность $|a_1^{\uparrow}| \leqslant m.$ Если верхнее множество $a_2^{\uparrow} = a_1^{\uparrow}$, то $a_2$ не включается в $A_0$. Иначе включается и $|a_1^{\uparrow} \cap a_2^{\uparrow}| \leqslant m-1$. Перебирая все элементы из $A$, множество $A_0^{\uparrow}$ стабилизируется ($0 \leqslant |A_0^{\uparrow}| \leqslant m$). Следовательно, верхний конус любого множества $A$ частичного порядка из $\K$ является собственным конечнопорожденным. Это доказывает нётеровость по уравнениям категории $\K$.
	
\end{document}
