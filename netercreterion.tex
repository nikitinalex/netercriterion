\documentclass[12pt]{article}
\usepackage[utf8]{inputenc}
\usepackage[russian]{babel}
\usepackage{amssymb,amsmath}
\usepackage[noend]{algorithmic}
\usepackage{tikz}  
\usepackage{tkz-berge}
\usepackage{multicol}
\usepackage{ntheorem}
\usepackage{lipsum}
\usetikzlibrary{graphs,arrows,shapes,positioning}
\usepackage{graphicx}
\textheight=24cm % высота текста
\textwidth=16cm % ширина текста
\oddsidemargin=0pt % отступ от левого края
\topmargin=-1.5cm % отступ от верхнего края
\parindent=24pt % абзацный отступ
\parskip=0pt % интервал между абзацами
\tolerance=2000 % терпимость к "жидким" строкам
\flushbottom % выравнивание высоты страниц
%\def\baselinestretch{1.5} % печать с большим интервалом

\theoremstyle{break}
\newtheorem{Defini}{Определение}
\newtheorem{Problem}{Задача}
\newtheorem{Th}{Теорема}
\newtheorem{Con}{Следствие}
\newtheorem{Lem}{Лемма}
\newenvironment{Proof} % имя окружения
	{\par\noindent{\bf Доказательство.}} % команды для \begin
	{\hfill$\scriptstyle\blacksquare$} % команды для \end
\def\P{\mathcal{P}}
\def\S{$S(X_n)$}
\def\K{$K_m$}
\def\B{$\mathcal{B}$}
\def\S{S_n}
\def\algorithmicrequire{\textbf{Вход:}}
\def\algorithmicensure{\textbf{Выход:}}

\newcommand{\LD}{\langle}
\newcommand{\RD}{\rangle}

\title{Критерий нётеровости для категорий частичного порядка}
\author{И. Кудык, А.Ю. Никитин}
%\date{}

\begin{document}
    \maketitle % вывести заголовок, автора, дату
    \thispagestyle{empty} % не нумеровать первую страницу
    \tableofcontents % сгенерировать оглавление
    \newpage
    
    \section{Введение} % первый раздел

	\section{Предварительные сведения}
	
	\section{Основной результат}
		Любое множество элементов $A$ частичного порядка $\P$ порождает верхний и нижний конус $A^{\uparrow}$ и $A^{\downarrow}$. Верхний конус множества $A$ называется \textit{конечнопорожденным}, если существует конечное множество элементов $B$ частичного порядка $\P$ такое, что $A^{\uparrow} = B^{\uparrow}$ (для нижнего конуса соответственно, $A^{\downarrow} = B^{\downarrow}$). Если, при этом, $B\subseteq A,$ то конус $A^{\uparrow}$ называется \textit{собственным}.

		\begin{Th}
			Категория частичных порядков $\mathcal{K}$ является нётеровой по уравнениям тогда и только тогда, когда для любого подмножества $A$ элементов частичного порядка из категории $\mathcal{K}$ конусы $A^{\uparrow}$ и $A^{\downarrow}$ являются собственными конечнопорожденными конусами.
		\end{Th}

		\begin{Proof}
			Сначала, пусть для частичного порядка $\P$ из категории $\mathcal{K}$ конусы любого подмножества его элементов являются собственными конечнопорожденными. Пусть задана система уравнений $S(X_n)$ над $\P$ в языке $L_A$ от $n$ переменных. Сразу предположим, что эта система содержит все 7 типов уравнений и уравнений каждого типа тоже бесконечное число.

			Без ограничения общности, считаем, что все различные константы в системе $S(X_n)$ имеют различные интерпретации на $\P$. Это условие не является обязательным, но упростит дальнейшие рассуждения и не повлияет на суть доказательства.

			Сначала рассмотрим подмножество системы уравнений $S_{a=a}\subset S(X_n)$. Если это множество уравнений совместно над $\P$, то оно никак не влияет на множество решений системы уравнений. Если же $\P\nvDash S_{a=a},$ тогда существует уравнение $a_i=a_j\in S_{a=a},$ которое не выполнено над частичным порядком $\P$ и это уравнение будет эквивалентно системе $S(X_n).$ Аналогичная процедура касается множества уравнений $S_{a\leqslant}\subset S(X_n).$

			В виду конечности множества переменных, можно выбрать конечные подмножества уравнений $S'_{x=x}$ и $S'_{x\leqslant x}$ , эквивалентных системам $S_{x=x}$ и $S_{x\leqslant x}$ соответственно. Отметим, что множества $S_{x=x}$ и $S_{x\leqslant x}$ не могут быть противоречивыми.

			Если множество уравнений $S_{x=a}\subset S(X_n)$ совместно, то, опять, в виду конечности множества переменных, существует конечная подсистема $S'_{x=a}\subset S_{x=a}$, эквивалентная $S_{x=a}.$ Иначе, в $S_{x=a}$  существует такая пара уравнений $x_i=a_j, x_i=a_k,$ что $a_j\neq a_k$. Эта пара уравнений и будет эквивалентной $S(X_n)$ подсистемой.

			Отметим, что до сих пор мы не воспользовались условиями из формулировки критерия. Осталось рассмотреть подмножества уравнений $S_{x\leqslant a}\subset S(X_n)$ и $S_{a\leqslant x}\subset S_{X_n}$. Сначала рассмотрим $S_{x\leqslant a}$.

			Выделим подмножество уравнений, зависящих только от одной переменной $S_{x_i\leqslant a}\subset S_{x\leqslant a}$. Это множество можно представить следующим образом: $S_{x_i\leqslant a}=\{x_i\leqslant a_j\ |\ j\in J, |J| = \infty\}.$ Из этого представления видно, что решение системы уравнений $S_{x_i\leqslant a}$ принадлежит конусу $A_i^{\downarrow}$, где $A_i = \{a_j\ |\ j\in J\}$. Пользуясь свойством, что конусы для любого множества элементов частичного порядка собственные конечнопорожденные, можно выбрать конечное подмножество $J'\subset J$ такое, что $A'_i = \{a_j\ |\ j\in J'\}$ и $(A'_i)^{\downarrow} = A_i^{\downarrow}.$ Это означает, что можно выбрать конечную подсистему $S'_{x_i\leqslant a}\equiv S_{x_i\leqslant a}$.

			Применяя описанную процедуру для всех переменных из $S_{x\leqslant a}$, можно выделить конечную подсистему $S'_{x\leqslant a}\equiv S_{x\leqslant a}.$ Аналогичными рассуждениями, можно вывести существование конечной подсистемы $S'_{a\leqslant x}\equiv S_{a\leqslant x}.$

			Тем самым показано, что если верхний и нижний конусы любого подмножества элементов частичного порядка собственные конечнопорожденные, то любой бесконечной системе $S(X_n)$ эквивалентная конечная подсистема $S'_{x=x}\cup S'_{x\leqslant x}\cup S'_{x=a}\cup S'_{x\leqslant a}\cup S'_{a\leqslant x}.$  

			Теперь, пусть категория частичных порядков $\mathcal{K}$ является нётеровой по уравнениям. Выберем произвольное бесконечное множество $A$ частичного порядка $\P$ из категории $\mathcal{K}.$ Рассмотрим его нижний конус $A^{\downarrow}$. Составим систему уравнений $S(x) = \{x\leqslant a_i\ |\ a_i\in A\}$ над $\P$. В виду того, что категория $\mathcal{K}$ нётерова по уравнениям, можно выбрать конечную подсистему $S'(x)=\{x\leqslant a_i\ |\ a_i\in A', |A'|<\infty\}.$ Это означает, что $A^{\downarrow} = A'^{\downarrow},$ что доказывает собственную конечнопорожденность нижнего конуса для любого множества элементов частичного порядка. Аналогично доказывается собственная конечнопорожденность верхних конусов.

		\end{Proof}

		Отметим, что не любой конечнопорожденный конус является собственным. (Тут привести контрпример)

		В виду этого замечания и доказанного критерия нётеровости для категорий частичного порядка, можно сформулировать следующий критерий слабой нётеровости.

		\begin{Con}
			Категория частичных порядков $\mathcal{K}$ обладает свойством слабой нётеровости по уравнениям тогда и только тогда, когда для любого подмножества $A$ элементов частичного порядка из категории $\mathcal{K}$ конусы $A^{\uparrow}$ и $A^{\downarrow}$ являются конечнопорожденными.
		\end{Con}


	\section{Примеры классов частичных порядков}
		
	
\end{document}
